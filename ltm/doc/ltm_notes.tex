\documentclass[english]{article}
\usepackage[T1]{fontenc}
\usepackage[latin9]{inputenc}
\usepackage{geometry}
\geometry{verbose,tmargin=0.8in,bmargin=0.8in,lmargin=0.7in,rmargin=0.7in}
\usepackage{graphicx}
\usepackage{amsmath}
\usepackage{amsfonts}
\usepackage{babel}
\usepackage{algorithmic}
\usepackage[ruled]{algorithm2e}
\usepackage{amsthm}
\usepackage{enumerate}


\DeclareMathOperator*{\argmin}{\arg\!\min}
\DeclareMathOperator*{\argmax}{\arg\!\max}

\begin{document}

\newcounter{qcounter}

\date{}
\title{\textbf{Representations for Manipulation of Articulated Objects}}
%\author{Venkatraman Narayanan}
\maketitle
\section*{Terminology and Notation}
The deformable model (d-model) is a graph $G=(V,E)$ with vertices $v\in SE(3)$ and the edges
representing the kinematic constraint between the vertices. A vertex $v$ is defined by the
position of the point $x$ and its rotation $R$ with respect to the world frame.
For vertices $v_i, v_j\in V$, the correspoding edge $e_{ij}=(v_i,v_j)$ is associated with three parameters $(\tau_{ij}, c_{ij}, r_{ij})$ defined by:
\begin{align*}
  \tau_{ij} &= \begin{cases}
  0 \mbox{ if the joint is rigid}\\
  1 \mbox{ if the joint is prismatic} \\
  2 \mbox{ if the joint is revolute} \\
  3 \mbox{ if the joint is planar (no rotation)} \\
  4 \mbox{ if the joint is cylindrical} \\
  5 \mbox{ if the joint is planar} \\
  6 \mbox{ if the joint is spherical} \\
  7 \mbox{ if the joint is a helix} \\
\end{cases}
\end{align*}
\subsection*{Rigid joint}
 $c_{ij}$ and $r_{ij}$ are undefined.
\subsection*{Prismatic joint}
$c_{ij}$ is the direction of motion defined by the prismatic joint. $r_{ij}$ is undefined.
\subsection*{Revolute joint}
$c_{ij}$ is the axis of rotation and $r_{ij}$ is the radius of rotation.
\subsection*{Planar joint (no rotation)}
$c_{ij}$ is the normal to the plane of motion.
\subsection*{Spherical joint}
\end{document}
