\documentclass[english]{article}
\usepackage[T1]{fontenc}
\usepackage[latin9]{inputenc}
\usepackage{geometry}
\geometry{verbose,tmargin=0.8in,bmargin=0.8in,lmargin=0.7in,rmargin=0.7in}
\usepackage{graphicx}
\usepackage{amsmath}
\usepackage{amsfonts}
\usepackage{babel}
\usepackage{algorithmic}
\usepackage[ruled]{algorithm2e}
\usepackage{amsthm}
\usepackage{enumerate}


\DeclareMathOperator*{\argmin}{\arg\!\min}
\DeclareMathOperator*{\argmax}{\arg\!\max}

\begin{document}

\newcounter{qcounter}

\date{}
\title{\textbf{Learning Articulation}}
%\author{Venkatraman Narayanan}
\maketitle
\section*{Learning Model Parameters}
Let $x_{ij}^{(t)}$ denote the position vector of $p_j$ in $p_i$'s local coordinate frame. Then,
\begin{align*}
  a_{ij} = \frac{1}{T}\sum_{i=1}^T x_{ij}^{(t)}\\
  c_{ij}^{prismatic} = \frac{a_{ij}}{\|a_{ij}\|}\\
  r_{ij} = \frac{1}{T}\sum_{i=1}^T \|x_{ij}^{(t)}\|\\
\end{align*}
TODO: Transform $\delta$ to reflect noise in the angle.
Let $\delta$ be the observation noise variance. Then the likelihood
\begin{align*}
  P(x_{ij}^{(t)}|\text{prismatic}) &=  \mathcal{N}(x_{ij}^{(t)}\cdot c_{ij}^{prismatic};0, \delta)\\
   P(x_{ij}^{(t)}|\text{revolute}) &=  \mathcal{N}(\|x_{ij}^{(t)}\|; r_{ij}, \delta)\\
\end{align*}
Assuming the observation probabilities of the edges at each timestep are independent given the joint types, we have:
\begin{align*}
  P(x_{ij}|\text{prismatic}) = \prod_{t=1}^T P(x_{ij}^{(t)}|\text{prismatic})\\
  P(x_{ij}|\text{revolute}) = \prod_{t=1}^T P(x_{ij}^{(t)}|\text{revolute})\\
\end{align*}

TODO: penalize for model complexity

  
\end{document}
